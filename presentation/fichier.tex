\documentclass{beamer}
\usepackage[french]{babel}
\usepackage[utf8]{inputenc}
\usepackage{amsmath}
\usepackage{multicol}
\title{Projet de physique Numérique - Allée de Von Karman}
\subtitle{Simulation python}
\author[Cadiou Corentin \and Petit Antoine]
{Cadiou Corentin \and Petit Antoine}
\date{Janvier 2014}
\subject{Physique numérique}

\AtBeginSubsection[]
{
  \begin{frame}
    \frametitle{Table of Contents}
    \tableofcontents[currentsubsection]
  \end{frame}
}

\renewcommand{\O}{\mathcal{O}}
\newcommand{\vect}[1]{\boldsymbol{#1}}
\renewcommand{\d}{\textrm{d}}
\newcommand{\nablaf}{\overrightarrow\nabla}
\begin{document}
  \frame{\titlepage}

  \section{Mise en place de la simulation}
  \subsection{Point de départ}
  \begin{frame}
    \frametitle{Point de départ}
    Code initial : simulation d'une cellule de Rayleigh-Benard.

    Schéma d'advection :
    \begin{itemize}
      \item advection 1/2 lagrangienne $\rightarrow$ diffusion (u,v);
      \item advection 1/2 lagrangienne $\rightarrow$ diffusion (T);
    \end{itemize}
  \end{frame}
  \begin{frame}
    \frametitle{Point de départ}
    Premières modifications :
    \begin{itemize}
      \item<1-> suppression de la température;
      \item<2-> modification des conditions aux limites:
        \begin{itemize}
          \item<3-> haut et bas : 
            \[ \left. \frac{\partial u}{\partial y}\right|_{y=0,H} = 
            \left. \frac{\partial v}{\partial y}\right|_{y=0,H} = 0 \] 
          \item<4-> droite :
            \[ \left. \frac{\partial u}{\partial x}\right|_{x=W} = 
            \left. \frac{\partial v}{\partial y}\right|_{y=W} = 0 \]
          \item<5-> gauche :
            \[ \left. \frac{\partial u}{\partial x}\right|_{x=0} = u_0
            \qquad \left. \frac{\partial v}{\partial x}\right|_{x=0} = 0\]
        \end{itemize}
    \end{itemize}
  \end{frame}

  \subsection{Incorporation du problème}
  \begin{frame}
    \frametitle{Obstacle}
    Pour commencer, nous avons imposé un obstacle fixe (noté $\O$) :
    \[ \forall (x,y) \in \O : u(x,y) = v(x,y) = 0 \]
    On visualisait en regardant le champ de vitesse \dots
    \begin{center}
      \includegraphics[width=0.6\textwidth]{quiver.png}
    \end{center}
  \end{frame}
  \begin{frame}
    \frametitle{Traceurs}
    \dots avant de remplacer cet affichage par la visualisation de
    traceurs, c'est-à-dire un champ supplémentaire noté $T$ advecté et
    passif tel que :
    \begin{align*}
      \forall n\in \mathbb{Z}\quad T(0,n \Delta, t>0) & = 1\\
      \forall (x,y)\quad T(x,y,0) & = 0
    \end{align*}
    \begin{center}
      \includegraphics[height=0.6\textheight]{tracer0.png}
    \end{center}
    Ce n'est pas satisfaisant pour observer les tourbillons.
  \end{frame}
  
  \begin{frame}
    \frametitle{Traceurs ``derrière''}
    Idée : l'obstacle impose $\forall (x,y) \in \O: T^n(x,y) = 0$ et on met
    initialement  $\forall (x,y) \not\in \O : T^0 = 1$.
    \begin{center}
      \includegraphics[height=0.6\textheight]{tracer1.png}
    \end{center}
  \end{frame}
  \subsection{BFECC}
  \begin{frame}
    \frametitle{Back and Forth Error Compensation and Correct }
    On effectue les pas suivants pour un champ scalaire $X^n(x,y)$, un
    champ de vitesse $\vect{v}$ et un opérateur d'advection
    $A(\vect{v},X)$ :
    \begin{align}
      X_1 & = A(\vect{v},X^n) \\
      X_2 & = A(-\vect{v},X_1) \\
      X^{n+1} & = A(\vect{v},X - \alpha(X^n - X_2))
    \end{align}
    Dans la littérature, $\alpha = \frac{1}{2}$.
  \end{frame}

  \begin{frame}
    \frametitle{BFECC instable}
    Dans nos conditions (pas de diffusion physique, haut Reynolds)
    : BFECC instable
    \begin{center}
      \includegraphics[height=0.8\textheight]{instab_BFECC.png}
    \end{center}
  \end{frame}
  \begin{frame}
    \frametitle{Solution : diminuer $\alpha$}
    Si on diminue $\alpha$, la diffusion numérique augmente :
    \centering \includegraphics[width=1.\textwidth]{comparaison_alpha.png}
  \end{frame}
  
  \subsection{L'obstacle qui oscille}
  \begin{frame}
    \frametitle{Idée naïve}
    On se contente de définir $\O(t) = T_{\vect{v(t)}}\ \O(0)$ avec 
    \begin{itemize}
      \item $T_{\vect{v}}$ l'opérateur de translation ;
      \item $\vect{v}$ la vitesse de l'obstacle, par exemple $\sin(\omega
        t)\vect{u_y}$.
    \end{itemize}
    \bigskip
   On impose pour $\phi^n$ un champ :
    \[ \forall (x,y)\in\O \quad \phi^n(x,y) = 0 \]
  \end{frame}
  \begin{frame}
    \frametitle{Deux problèmes}
    Discontinuités de la force de trainée.
    \begin{center}
      \includegraphics[height=0.7\textheight]{penalisation.png}
    \end{center}
  \end{frame}

  \begin{frame}
    \frametitle{Deux problèmes}
    ``Absorption'' des vecteurs par l'obstacle.
    \begin{center}
      \includegraphics[height=0.7\textheight]{absorption.png}
    \end{center}
  \end{frame}

  \begin{frame}
    \frametitle{Pénalisation}
    \begin{itemize}
    \item $\O^n$ l'obstacle à l'instant $n$ ;
    \item $\vect{u}(x,y)$ la vitesse de l'obstacle au point $(x,y)$ ;
    \item $\vect{v}^n$ le champ de vitesse du fluide au temps $n$;
    \end{itemize}
    \bigskip
    On impose :
    \[ \forall (x,y) \in \O^n \quad \vect{v}^{n+1} = \vect{u} + 
    (\vect{v}^n - \vect{u}) \cdot e^{-\beta \mathrm{d} t} \]
  \end{frame}

  \begin{frame}
    \frametitle{Pénalisation}
    \begin{center}
      \includegraphics[width=0.33\textwidth]{penalisation_10.png}
      \includegraphics[width=0.33\textwidth]{penalisation_100.png}
      \includegraphics[width=0.33\textwidth]{penalisation_1000.png}
    \end{center}
  \end{frame}
  
\section{Le schéma}

\subsection{L'advection}
\begin{frame}
  \frametitle{L'advection}
    Le schéma de l'étape de diffusion pour la composante $u$ :
  \[{u^*}_{i \ j}^{n} = u_{i \ j}^n - {\d t}\left( u_{ij}^n \frac{u_{i+1
      \ j}^n-u_{i-1 \ j}^n}{2 \d x} + v_{i \ j}^{n} \frac{u_{i \
      j+1}^n-u_{i \ j-1}^n}{2 \d y}\right)
  \]
  \end{frame}		
  
  \subsection{La diffusion} 
  \begin{frame}
    \frametitle{La diffusion}
    Le schéma de l'étape de diffusion pour la composante $u$ : 
    \begin{align*}
    {u^{**}}_{i j}^n = {u^*}_{i \ j}^{n} + \frac{\d t}{\mathrm{Re}}
    \Bigg( & \frac{{u^*}_{i+1 \ j}^n + {u^*}_{i-1 \ j}^n}{{\d x}^2}
    + \frac{{u^*}_{i \ j+1}^n + {u^*}_{i \ j-1}^n}{{\d y}^2}\\
    &  -\frac{{u^*}_{i \ j}^{n}}{2({\d x}^2 + {\d y}^2)} \Bigg)
    \end{align*}
  \end{frame}	
  
  \subsection{La projection}
  \begin{frame}
    \frametitle{La projection}
    À cette étape, $\overrightarrow{u^{**}}$ vérifie seulement l'équation partielle
    \[ \frac{\partial\overrightarrow{u}}{\partial t} +
    (\overrightarrow{u} \cdot  \nablaf) \overrightarrow{u} =
    \frac{1}{\mathrm{Re}} \overrightarrow{\Delta} \overrightarrow{u} \]
    c'est-à-dire sans le gradient de la pression, en conséquence,
    $\overrightarrow{u}$ n'est pas de divergence nulle :
    \begin{align*}
      \nablaf \cdot \overrightarrow{u^{**}} & = \nablaf \cdot
      \overrightarrow{u}	& + &\nablaf \cdot ( [\mathrm{Schema 
        \ advection}] + [\mathrm{Schema \ diffusion}])\\ 
      & = 0  &+ & (\nablaf \cdot \nablaf) \phi \\ 
      & = \Delta \phi & & \\
    \end{align*}
  \end{frame}	 
  
\section{Résultats de la simulation}

 \subsection{L'allée de Von Karman}
 \begin{frame}
   \frametitle{L'allée de Von Karman}
   \centering \includegraphics[height= 0.7 \textheight]{VK_pas_mal.png}	
 \end{frame}
 
 \subsection{La propulsion en fonction de l'amplitude}
 		
 \begin{frame}
   \frametitle{L'effet de l'amplitude d'oscillation}
   On fait varier l'amplitude d'oscillation
   \centering \includegraphics[height= 0.6 \textheight]{9courbes.png}\\
   Travail de la force de traînée pour une amplitude entre 0,01\degre et 0,09\degre à une fréquence de 10 Hz.
 \end{frame}
 	
 \begin{frame}
   \frametitle{L'effet de l'amplitude d'oscillation}
   Sur une plus grande plage d'amplitudes, la traînée devient motrice
   \centering \includegraphics[width= 0.8 \linewidth]{bcp_points.png}\\
   Travail de la traînée au bout de 2000 itérations
 \end{frame}
  
 \begin{frame}
   \frametitle{Interpolation}
   On remarque que pour $ \theta_0 < 0,01$, $W_t \propto \theta_0^{-1}$
   \centering \includegraphics[width= 0.8 \linewidth]{9_courbes_extraites.png}\\
   Régression linéaire de $W_t^{-1}$ en fonction de $\theta_0$
 \end{frame}

 \subsection{La propulsion en fonction de la fréquence}
 	
 \begin{frame}
   \frametitle{L'effet de la fréquence d'oscillation}
   On fait varier la fréquence d'oscillation
   \centering \includegraphics[width= 0.8 \linewidth]{freq0,02.png}\\
   Travail de la force de traînée pour une fréquence entre 10 Hz et 20
   Hz à une amplitude de 0,02\degre. 
 \end{frame}
 	
 \begin{frame}
   \frametitle{Interpolation}
   On interpole la force de traînée moyenne (pentes des droites
   d'interpolations des droites précédentes) 
   \centering \includegraphics[width= 0.8 \linewidth]{modulationfreq.png} \\
   La force de traînée est linéaire en fréquence sur la plage étudiée.
 \end{frame}
\end{document}