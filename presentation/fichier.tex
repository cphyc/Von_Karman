\documentclass{beamer}
\usepackage[french]{babel}
\usepackage[utf8]{inputenc}
\title{Projet de physique Numérique - Allée de Von Karman}
\subtitle{Simulation python}
\author[Cadiou Corentin \and Petit Antoine] % (optional, for multiple authors)
{Cadiou Corentin \and Petit Antoine}
\date{Janvier 2014}
\subject{Physique numérique}

\AtBeginSection[]
{
  \begin{frame}
    \frametitle{Table of Contents}
    \tableofcontents[currentsubsection]
  \end{frame}
}

\renewcommand\O{\mathcal{O}}
\begin{document}
  \frame{\titlepage}

  \section{Historique}
  \subsection{Point de départ}
  \begin{frame}
    \frametitle{Point de départ}
    Code initial : simulation d'une cellule de Rayleigh-Benard.

    Schéma d'advection :
    \begin{itemize}
      \item advection 1/2 lagrangienne -> diffusion (u,v);
      \item advection 1/2 lagrangienne -> diffusion (T);
    \end{itemize}
  \end{frame}
  \begin{frame}
    \frametitle{Point de départ}
    Premières modifications :
    \begin{itemize}
      \item<1-> suppression de la température;
      \item<2-> modification des conditions aux limites:
        \begin{itemize}
          \item<3-> haut et bas : 
            \[ \left. \frac{\partial u}{\partial y}\right|_{y=0,H} = 
            \left. \frac{\partial v}{\partial y}\right|_{y=0,H} = 0 \] 
          \item<4-> droite :
            \[ \left. \frac{\partial u}{\partial x}\right|_{x=W} = 
            \left. \frac{\partial v}{\partial y}\right|_{y=W} = 0 \]
          \item<5-> gauche :
            \[ \left. \frac{\partial u}{\partial x}\right|_{x=0} = u_0
            \qquad \left. \frac{\partial v}{\partial x}\right|_{x=0} = 0\]
        \end{itemize}
    \end{itemize}
  \end{frame}

  \subsection{Incorporation du problème}
  \begin{frame}
    \frametitle{Obstacle}
    Pour commencer, nous avons imposé un obstacle fixe (noté $\O$) :
    \[ \forall (x,y) \in \O : u(x,y) = v(x,y) = 0 \]
    On visualisait en regardant le champ de vitesse \dots
    \begin{center}
      \includegraphics[height=0.7\textheight]{quiver.png}
    \end{center}
  \end{frame}
  \begin{frame}
    \frametitle{Traceurs}
    \dots avant de remplacer cet affichage par la visualisation de
    traceurs, c'est-à-dire un champ supplémentaire noté $T$ \emph{passif
    mais advecté}.
    \begin{center}
      \includegraphics[height=0.7\textheight]{tracer0.png}
    \end{center}
    Ce n'est pas satisfaisant pour observer les tourbillons.

  \end{frame}
  
\end{document}